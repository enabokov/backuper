% зазначаємо стильовий файл, який будемо використовувати
\documentclass{bachelor}

\usepackage{verbatim}

\newcommand{\bigG}{\mathbb{G}}
\newcommand{\inR}{\in_R}
\newcommand{\Mod}[1]{\ (\text{mod}\ #1)}

\renewcommand{\floatpagefraction}{.8}
\DeclareMathOperator{\ind}{ind}

% починаємо верстку документа
\begin{document}
    \maketitlepage{
        StudentName={Набоков Едуард Максимович}, 
        StudentMale=true,
        StudentGroup={ІП-52},
        ThesisTitle={Автоматизована централізована система бекапу баз данних},
        Advisor={доцент, канд. техн. наук, доцент Гавриленко~О.~В.},
        Reviewer={проф., д-р техн. наук, проф. Сидоренко~С.~С.},
        Year={2016}
    }

    \assignment{
        StudentName={Набоков Едуард Максимович},
        StudentMale=true,
        ThesisTitle={Автоматизована централізована система бекапу баз данних},
        Advisor={Гавриленко О.В., канд. техн. наук, доцент},
        Order={\invcommas{28}~травня~2018~р.~\No~000-C},
        ApplicationDate={\invcommas{15}~червня~2018~р.},
        InputData={реалізована система повинна створювати повні копії будь-яких баз данних через единий інтерфейс та зберігати у хмарних платформах за заданим користувачем інтервалом},
        Contents={проаналізувати методи створення бекапу баз данних, обрати швидкий та відмовостійкий метод створення резервних копій, спроектувати та реалізувати мікросервісну архітектуру, забезпечити універсальний підхід, надійність та цілістність трансферу данних для кожної з баз данних},
        Graphics={схема мікросервісної архітектури},
        ConsultantChapter={Розділ 3. Математичне забезпечення},
        Consultant={Бондаренко~Б.~Б., старший~викладач},
        AssignmentDate={\invcommas{15}~квітня~2018~р.},
        Calendar={
        1 & Перегляд та збір джерел за данною темою & 12.11.2018 & \\
        \hline
        2 & Аналіз та оцінювання методів створення резервених копій для кожної з баз данних & 14.12.2018 & \\
        \hline
        3 & Аналіз та оцінка REST та RPC підходів для реалізування мікросервісної архітектури & 24.12.2018 & \\
        \hline
        4 & Підготовка матеріалів першого розділу роботи & 01.02.2019 & \\
        \hline
        5 & Реалізація архітектури, створення бекапу hbase, зберігання у AWS S3 & 01.03.2019 & \\
        \hline
        6 & Підготовка матеріалів другого розділу роботи & 15.03.2019 & \\
        \hline
        7 & Підготовка матеріалів третього розділу роботи & 05.04.2019 & \\
        \hline
        8 & Підготовка матеріалів четвертого розділу роботи & 03.05.2019 & \\
        \hline
        9 & Оформлення пояснювальної записки & 01.06.2019 & \\
        },
        StudentNameShort={Набоков~Е.~М.},
        AdvisorNameShort={Гавриленко~О.~В.},
        Year={2019}
    }

    %!TEX root = ../abstract.tex
% створюємо анотацію
% \setcounter{page}{4}
\chapter*{Анотація}
\pagestyle{empty}
\setfontsize{14}
\thispagestyle{empty}
% далі пишемо текст анотації

% анотація повинна починатися інформацією про структуру роботи
% (кількість аркушів (БЕЗ ДОДАТКІВ!), додатків, посилань, рисунків і таблиць)
Роботу виконано на 42 аркушах, вона містить перелік посилань на використані джерела з $42$ найменувань.

% далі потрібно вказати мету роботи
\textbf{Метою} даної дипломної роботи ииии побудова програмного комплексу для розпізнавання промовленного тексту за відеорядом міміки обляччя людини.

\textbf{Об'єктом дослідження} є something.

\textbf{Предметом дослідження} є something.

Lorem ipsum dolor sit amet, consectetur adipiscing elit. Vivamus malesuada sapien mattis justo pellentesque commodo. Nulla aliquet lorem nec dolor pellentesque, rhoncus vestibulum velit auctor. Suspendisse potenti. Vivamus at sapien velit. Maecenas rhoncus egestas purus sed cursus. Fusce posuere nisl quis sem laoreet faucibus. Ut cursus at libero et iaculis. Donec finibus, nunc sit amet cursus lobortis, turpis lectus consectetur neque, non accumsan felis diam eget lorem. Mauris auctor dolor arcu, a aliquam diam vestibulum ut. Maecenas a cursus lectus, egestas eleifend magna. 
 % наприкінці анотації потрібно зазначити ключові слова
\MakeUppercase{Lorem, ipsum, dolor, sit, amet}

% створюємо анотацію англійською мовою
\chapter*{Abstract}
\thispagestyle{empty}
% анотація повинна починатися інформацією про структуру роботи
% (кількість аркушів (БЕЗ ДОДАТКІВ!), додатків, посилань, рисунків і таблиць)
The thesis is presented in N pages. It contains bibliography of N references.

% % далі потрібно вказати мету роботи

The \textbf{goal} Lorem ipsum dolor sit amet, consectetur

\textbf{The object} Lorem ipsum dolor sit amet, consectetur

\textbf{The subject} Lorem ipsum dolor sit amet, consectetur


% % далі потрібно вказати розглянуті методи та критерії, за яким вибрано один із них

Lorem ipsum dolor sit amet, consectetur adipiscing elit. Vivamus malesuada sapien mattis justo pellentesque commodo. Nulla aliquet lorem nec dolor pellentesque, rhoncus vestibulum velit auctor. Suspendisse potenti. Vivamus at sapien velit. Maecenas rhoncus egestas purus sed cursus. Fusce posuere nisl quis sem laoreet faucibus. Ut cursus at libero et iaculis. Donec finibus, nunc sit amet cursus lobortis, turpis lectus consectetur neque, non accumsan felis diam eget lorem. Mauris auctor dolor arcu, a aliquam diam vestibulum ut. Maecenas a cursus lectus, egestas eleifend magna. 
 
% % далі потрібно коротко викласти суть роботи
% The selected  is implemented in parallel computation model and is executed on the selected  platform. From the results of this execution analysis of optimal algorithm parameters and approximate problem size solvable in the span of one calendar year were made.

% % далі потрібно подати відомості про апробацію роботи


% The results could be used for estimating attack cost on popular asymmetric .

% % наприкінці анотації потрібно зазначити ключові слова
\MakeUppercase{Lorem, ipsum, dolor, sit, amet}


    % створюємо зміст
    % \includepdf[pages={-}]{abstract.pdf}
    \pagenumbering{gobble}
    \tableofcontents
    \cleardoublepage
    \pagenumbering{arabic}
    \setcounter{page}{8}

    % створюємо перелік умовних позначень, скорочень і термінів
    %!TEX root = ../thesis.tex
% створюємо перелік умовних позначень, скорочень і термінів
\shortings

\textbf{RPC} (remote procedure call), \textbf{REST} (representational state transfer) --- два механізми спілкування сервісів у мережі.

\textbf{gRPC} --- реалізація RPC мовою Golang.

\textbf{protobuf} --- швидкий протокол спілкування сервісів у RPC.

\textbf{goroutines} --- програмний потік, який керується runtime-ом мови Go. Може уснівати декілька в межах одного операційного потоку.

\textbf{GOMAXPROX} --- максимальна кількісті створення операційних потоків мовою Go.

\textbf{HDFS} (hadoop distributed filesystem) --- розподілена файлова система. Реалізована на Java.

\textbf{HBase} --- нереляційна колоночна база данних від Apache. Існує у фреймворку/екосистемі Hadoop. Реалізована на Java. Працює поверх HDFS.

\textbf{splice} --- метод проксіювання данних через операційний pipe не зберігаючи при цьому короткочасні дані у user space.

\textbf{cron job} --- процес, який виконується у фоні та не блокує основний процес виконнання програмного забезпечення.

    %!TEX root = ../thesis.tex
% створюємо вступ
\intro

Сьогодні всі high-load системи розгортаються у хмарних платформах --- AWS, GCP, Azure, IBM, Digital Ocean, Yandex \cite{clouds2017forbes}.
Задля забезпечення fault та partition toleranсe потребується більше серверів, дата-центрів.
Вони розміщаються у різних регіонах Землі. Інфраструктура стає громіздкою, ймовірність помилки зростає,
контролювати стає складніше. Потребується багато зусиль та ресурсів. У світі системного та девопс
адміністрування обов'язковим є налагодження централізованої системи моніторингу та логування.
Ці системи допомагають виявляти проблемні місця у таких громіздких high-load інфраструктурах
та нотифікують системних адміністраторів у разі її появи або спрогнозування проблеми.

Сучасний світ - інформаційний. Практично кожен проект зчитує, обробляє/трансформує, зберігає або перенаправляє інформацію.
Вона може бути різною за значення та вагою. Демографічна статистика, історія пошуку в браузері, банківські рахунки користувачів.
Для збереження будь-якої інформації викорстовуються бази данних. Для забезпечення надійності збереження данних --- резервні копії \cite{sqlserver2012backup}.

На цей рік кількість баз данних нараховує більше 30 популярних \cite{topdb2019list} та 100 неофіційних.
Кожна має свої інструменти для управління. Бази данних поділяються за концепцією, 
масштабуванням, варіантами зберігання данних. SQL та NoSQL. Column, document,
graph та key \& value based \cite{nosql2018features}. Кожна вирішує певні задачі. 
Складно інтегрувати нову БД, так як виникає проблема у автоматизації управління БД.

Цей проект вирішує головну проблему з великою кількістю БД --- автоматизований централізований єдиний універсальний
інтерфейс для взаємодії з backup/restore будь-якої бази данних по всій інфраструктурі.

Окрім цього, користувач отримує:

\begin{itemize}
    \item інструмент до 50МБ.
    \item управління резервними копіями за допомогою декілької клавіш
    \item моніторинг бекапу баз данних
    \item заплановані цілодобові бекапи
    \item cloud агностик \cite{cloudagnostic2012oreilly}
    \item кросплатформність --- Linux, Unix, Windows
    \item взаємодія з Amazon Web Services, Google Cloud Platform, Microsoft Azure, International Business Machines Cloud, Digital Ocean, Yandex Cloud.
    \item нотифікації в Email, Slack
    \item можливість запускати в Kubernetes, Docker
    \item можливість інтегрувати в Internet of Things \cite{iot2015mit}.
\end{itemize}

    %!TEX root = ../thesis.tex

\chapter{Аналіз та оцінка методів створення резервних копій баз данних}

Перша та основна база данних для цього проекту --- \textit{hbase}. В цьому розділі буде розкрито декілька варіантів його бекапу.

\section{Export}
Вбудований інструмент для виконнання повної резервної копії таблиць hbase на файлову систему --- hdfs.
Паралельне копіювання виконується завдяки праці MapReduce.

Переваги:

\begin{itemize}
    \item паралельність
    \item нативний інструмент hbase
\end{itemize}

Недоліки:

\begin{itemize}
    \item повільний, якщо мережа з низькою пропускною здатністю
    \item потребую вільну память на hdfs для експортованих таблиць
    \item існування проміжного пункту - hdfs
\end{itemize}

\section{CopyTable}
Цей метод має такі самі переваги, як і у метода Export \cite{hbase2012backup}. CopyTable запускає також MapReduce задачу.
Відрізняється він тим, що копіює дані з таблиць hbase не на hdfs, а одразу в іншу таблицю, 
інший namespace, але в тому самому кластері. Одним із найголовніших недоліків даного методу є те, що CopyTable
копіює row-by-row таблиць та виконує put для кожного з них. 
При великих розмірах таблиці, CopyTable спричинює швидке наповнення MemStore та виклик Minor/Major Compaction.
Такий метод не рекомендується використовувати на великих данних.

\section{HTable API}
Корисний тільки для Java програмістів \cite{hbase2016htable}.
Цей метод дозволяє написати свою реалізацію бекапу.
І одразу виникає дві проблеми:

\begin{itemize}
    \item програміст повинен знати як правильно бекапити
    \item неможливість реалізувати на іншій мові програмування
\end{itemize}

\section{Offline backup}
Практичний метод, який дозволяє виконати повну резервну копію інформації.
HBase працює тільки поверх hdfs і зберігає там свої дані.
Цей метод пропонує копіювати файлову систему hdfs і її директорію /hbase/.
Задля забезпечення цілістних данних - hbase кластер повинен зупинити.
У разі копіювання одного чи декількох namespaces-ів --- усі їх таблиці повинні бути вимкнуті (disable) для забезпечення persistence.

\begin{comment}
LipNet is the first end-to-end model that performs sentence-level sequence prediction for visual speech recogntion. That is, we 
demonstrate the first work that takes as input as sequence of images and outputs a distribution over sequences of tokens; it is 
trained end-to-end using CTC and thus also does not require alignments.
\end{comment}

    
\chapter{Обґрунтування вибору засобів реалізації}

Данний розділ проаналізує та доведе вибір мови программування для реалізації мікросервісної архітектури 
та модулів для взаємодії з namenode hdfs-у.

\section{Мова программування}

Необхідні характеристики мови:

\begin{enumerate}
    \item Статична типізація
    \item Швидке компілювання та виконання операція
    \item Швидкий, вбудований garbage collector, мінімальний memory leak.
    \item Наявність необхідних реалізацій клієнтів hdfs, postgres та інших мов програмування.
    \item Підтримка паралельних обчислень
    \item Реалізація RPC протоколу
    \item Реалізація HTTP протоколу
\end{enumerate}

    \subsection{С/С++}

    Базова мова программування, яка породила багато інших мов --- Python, Java, Go, Perl, C\#, Dart \cite{lang2019cfamily}.
    Являється найпопулярнішою мовою в IT сфері.

    \begin{enumerate}
        \item Мова С має статичну типізацію задля забезпечення надійності програми та її коректного виконання \cite{clang2005intro,clang2005dynamic}.
        \item Мова C та C++ - компільована мова программування \cite{compile2014c}.
        Компілювання складається з 4 фаз: лексичний, синтаксичний, семантичний парсинг, компіляція, збірка проекту, лінкування.
        Підтримує, як статичне, так і динамічне лінкування \cite{linking2018clang,linking2019briefly}.
        \item Мова C та С++ не має вбудованого garbage collector. Керування памятью повинен реалізовувати програміст.
        Це великий недолік у швидкій реалізації програми. Може спричинити великий memore leak \cite{clang2005memory}.
        Окрім цього, займе ще час на аналіз програми та на її покращення в управлінні паматью. 
        Так як мікросервіси будуть управляти данними підчас свого функціонування - це вплине на те, що можуть виникати
        суттєво великі footprint-и.
        \item На мові C вже реалізовано модуль --- libhdfs \cite{libhdfs2018}, який дозволяє спілкуватись з namenode hdfs та виконувати базові операції з hdfs.
    \end{enumerate}

    % \subsection{Python}
    % \begin{enumerate}
    %     \item
    % \end{enumerate}

    % \subsection{Golang}
    % \begin{enumerate}
    %     \item
    % \end{enumerate}

    \input{chapters/6_algorithms}
    

    %!TEX root = ../thesis.tex
% створюємо Висновки до всієї роботи
\conclusions

Lorem ipsum dolor sit amet, consectetur adipiscing elit. Vivamus malesuada sapien mattis justo pellentesque commodo. 
Nulla aliquet lorem nec dolor pellentesque, rhoncus vestibulum velit auctor. Suspendisse potenti. Vivamus at sapien velit.
 Maecenas rhoncus egestas purus sed cursus. Fusce posuere nisl quis sem laoreet faucibus. Ut cursus at libero et iaculis. 
 Donec finibus, nunc sit amet cursus lobortis, turpis lectus consectetur neque, non accumsan felis diam eget lorem. Mauris 
 auctor dolor arcu, a aliquam diam vestibulum ut. Maecenas a cursus lectus, egestas eleifend magna.


Доцільними напрямками подальшої роботи можуть бути:

\begin{itemize}
    \item Lorem ipsum dolor sit amet, consectetur
    \item Lorem ipsum dolor sit amet, consectetur
    \item Lorem ipsum dolor sit amet, consectetur
\end{itemize}


    \bibliographystyle{ugost2003}
    \bibliography{thesis}
    % % створюємо додатки
    % % перший додаток повинен містити лістинги розроблених програм
    % \append{Лістинги програм}

    % % кожний лістинг вставляється в додаток за допомогою спеціальної команди,
    % % перший аргумент якої --- це заголовок, який з'являтиметься в тексті,
    % % другий --- шлях до файлу з лістингом
    % \listing{presenter.h --- прототип пред'явника-вчителя}{SRC/Presenter/presenter.h}

    % \listing{presenter.cpp --- реалізація пред'явника-вчителя}{SRC/Presenter/presenter.cpp}

    % \input{chapters/appendix_firstrun}

    % % останній додаток повинен містити слайди пр\езентації доповіді на захисті дипломної роботи
    % \append{Ілюстративний матеріал}

    % \begin{figure}[!htp]%
    % 	\centering
    % 	\includegraphics[scale = 0.43]{PNG/slide1.png}%
    % 	\caption{Слайд 1}%
    % 	\label{fig:p1}%
    % \end{figure}

    % \begin{figure}[!htp]%
    % 	\centering
    % 	\includegraphics[scale = 0.455]{PNG/slide2.png}%
    % 	\caption{Слайд 2}%
    % 	\label{fig:p2}%
    % \end{figure}

\end{document}
