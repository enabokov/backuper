%!TEX root = ../thesis.tex
% створюємо вступ
\intro

Сьогодні всі high-load системи розгортаються у хмарних платформах --- AWS, GCP, Azure, IBM, Digital Ocean, Yandex \cite{clouds2017forbes}.
Задля забезпечення fault та partition toleranсe потребується більше серверів, дата-центрів.
Вони розміщаються у різних регіонах Землі. Інфраструктура стає громіздкою, ймовірність помилки зростає,
контролювати стає складніше. Потребується багато зусиль та ресурсів. У світі системного та девопс
адміністрування обов'язковим є налагодження централізованої системи моніторингу та логування.
Ці системи допомагають виявляти проблемні місця у таких громіздких high-load інфраструктурах
та нотифікують системних адміністраторів у разі її появи або спрогнозування проблеми.

Сучасний світ - інформаційний. Практично кожен проект зчитує, обробляє/трансформує, зберігає або перенаправляє інформацію.
Вона може бути різною за значення та вагою. Демографічна статистика, історія пошуку в браузері, банківські рахунки користувачів.
Для збереження будь-якої інформації викорстовуються бази данних. Для забезпечення надійності збереження данних --- резервні копії \cite{sqlserver2012backup}.

На цей рік кількість баз данних нараховує більше 30 популярних \cite{topdb2019list} та 100 неофіційних.
Кожна має свої інструменти для управління. Бази данних поділяються за концепцією, 
масштабуванням, варіантами зберігання данних. SQL та NoSQL. Column, document,
graph та key \& value based \cite{nosql2018features}. Кожна вирішує певні задачі. 
Складно інтегрувати нову БД, так як виникає проблема у автоматизації управління БД.

Цей проект вирішує головну проблему з великою кількістю БД --- автоматизований централізований єдиний універсальний
інтерфейс для взаємодії з backup/restore будь-якої бази данних по всій інфраструктурі.

Окрім цього, користувач отримує:

\begin{itemize}
    \item інструмент до 50МБ.
    \item управління резервними копіями за допомогою декілької клавіш
    \item моніторинг бекапу баз данних
    \item заплановані цілодобові бекапи
    \item cloud агностик \cite{cloudagnostic2012oreilly}
    \item кросплатформність --- Linux, Unix, Windows
    \item взаємодія з Amazon Web Services, Google Cloud Platform, Microsoft Azure, International Business Machines Cloud, Digital Ocean, Yandex Cloud.
    \item нотифікації в Email, Slack
    \item можливість запускати в Kubernetes, Docker
    \item можливість інтегрувати в Internet of Things \cite{iot2015mit}.
\end{itemize}
