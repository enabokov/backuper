
\chapter{Обґрунтування вибору засобів реалізації}

Данний розділ проаналізує та доведе вибір мови программування для реалізації мікросервісної архітектури 
та модулів для взаємодії з namenode hdfs-у.

\section{Мова программування}

Необхідні характеристики мови:

\begin{enumerate}
    \item Статична типізація
    \item Швидке компілювання та виконання операція
    \item Швидкий, вбудований garbage collector, мінімальний memory leak.
    \item Наявність необхідних реалізацій клієнтів hdfs, postgres та інших мов програмування.
    \item Підтримка паралельних обчислень
    \item Реалізація RPC протоколу
    \item Реалізація HTTP протоколу
\end{enumerate}

    \subsection{С/С++}

    Базова мова программування, яка породила багато інших мов --- Python, Java, Go, Perl, C\#, Dart \cite{lang2019cfamily}.
    Являється найпопулярнішою мовою в IT сфері.

    \begin{enumerate}
        \item Мова С має статичну типізацію задля забезпечення надійності програми та її коректного виконання \cite{clang2005intro,clang2005dynamic}.
        \item Мова C та C++ - компільована мова программування \cite{compile2014c}.
        Компілювання складається з 4 фаз: лексичний, синтаксичний, семантичний парсинг, компіляція, збірка проекту, лінкування.
        Підтримує, як статичне, так і динамічне лінкування \cite{linking2018clang,linking2019briefly}.
        \item Мова C та С++ не має вбудованого garbage collector. Керування памятью повинен реалізовувати програміст.
        Це великий недолік у швидкій реалізації програми. Може спричинити великий memore leak \cite{clang2005memory}.
        Окрім цього, займе ще час на аналіз програми та на її покращення в управлінні паматью. 
        Так як мікросервіси будуть управляти данними підчас свого функціонування - це вплине на те, що можуть виникати
        суттєво великі footprint-и.
        \item На мові C вже реалізовано модуль --- libhdfs \cite{libhdfs2018}, який дозволяє спілкуватись з namenode hdfs та виконувати базові операції з hdfs.
    \end{enumerate}

    % \subsection{Python}
    % \begin{enumerate}
    %     \item
    % \end{enumerate}

    % \subsection{Golang}
    % \begin{enumerate}
    %     \item
    % \end{enumerate}
