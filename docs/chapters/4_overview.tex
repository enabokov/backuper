%!TEX root = ../thesis.tex
\chapter{АНАЛІЗ ВИМОГ ДО ПРОГРАМНОГО ЗАБЕЗПЕЧЕННЯ}

\section{Загальні положення}
Кожна база данних реалізована за певною концепцією та мовою програмування.
Тому кожна з них має свій механізм створення резервних копій.

За напрямком:
\begin{itemize}
    \item реляційні;
    \item нереляційні.
\end{itemize}

За концепцією:
\begin{itemize}
    \item документний -- надає механізм для зберігання строкових файлів у базі данних, їх швидку оброботку та зручний пошук;
    \item ключ-значення (хеш-таблиці) -- як правило, використовується для кешування та service 
    discovery в інфраструктурі контейнерів, хостів;
    \item колоночний (альтернатива реляційним строковим таблицям) -- ідея заключається в тому,
    що користувачу чи аплікейшну не потрібно усі колонки, а тільки декілька з них.
    Тому у таких базах данних змінено механізм збереження записів,
    який дозволяє швидко зчитувати колонки;
    \item графовий -- абстракційна модель у вигляді графу. 
    Основна його задача - зберігати відносини між сущностями.
\end{itemize}

Існує 3 універсальних підходи до створенння бекапів:
\begin{itemize}
    \item копіювання файлової системи, на яку БД зберігає дані.
\end{itemize}

\section{Змістовний опис і аналіз предметної області}

\subsection{Причини створення резервних копій}

\subsection{Повне резервне копіювання баз данних, характеристики}
Такий метод повністю копіює усі файли бази данних незважаючи на те, що були вони скопійованими
та зберігає цю копію в іншому місці, яке незалежне від поточного.

Повна копія - найкращий захист данних, включаючи його швидкодію та простоту. У разі виникнення помилки - дозволить реставрувати базу данних у новому середовищі завдяки цій копії і
продовжить функціонувати та обслуговувати клієнтів. Такий тип бекапу називається рівнем 0.
Зважаючи на те, що сучасний світ - інформаційний. Інфраструктури підтримують петабайти данних.
Для повного бекапу такої бази данних потрібно забагато часу. Інформація за цей час може втратити свою користь.
Окрім цього, для цього потрібно багато ресурсів для оперування данними.

У Hadoop системі існує сервісна одиниця - MapReduce \cite{mapreduce2012}.
Її задача - виділення/зчитування, обробка, зберігання/трансфер в інше місце \cite{etl2017}.
ETL - базова концепція опрацювання данних в сфері "великих данних" \cite{bigdata2015}.
MapReduce розбиває датасет на частини (split), обробляє кожну з них та завантажує в інше сховище \cite{mapreduce2015cookbook}.

Переваги повних резервних копій баз данних:
\begin{itemize}
    \item у разі стихійного лиха - забезпечує надійне відновлення усіх втрачених данних;
    \item вся резервна копія зберігається в одному файлі на файловій системі.
\end{itemize}

Недоліки повних резервних копій баз данних:
\begin{itemize}
    \item у разі великих данних - потребує памяті вдвічі більше для копій;
    \item витрачає багато часу на створення повної копії;
    \item у разі нелегализованого доступу до копії з третьої сторони - спричинює витік усіх данних.
\end{itemize}

\subsection{Покрокове резервне копіювання, характеристики}
Ідея такого підходу полягає у збереження різниці між поточним станом бази данних та останньою резервною копією.
Для коректної роботи цього підходу потрібно створити для початку повну резерну копію.
Після цього створювати пошагові копію, які будуть зрівнюватись з попередньою копією.

З цього випливає, що у разі втрати одного з проміжних копій - відновлення повної бази данних неможливе \cite{backup2015incrdiff}.

Переваги підходу:
\begin{itemize}
    \item швидке створення резерних копій
    \item потребує мало памяті
    \item може бути запущений декілька разів підряд. Кожна покрокова копія - пункт відліку майбутньої копії.
\end{itemize}

Недоліки підходу:
\begin{itemize}
    \item повільне повне відновлення резервної копії
    \item увесь ланцюг з копій повинен бути присутнім при повному відновленні. Інакше відновлення неможливе.
\end{itemize}

\subsection{Диференціальне резервне копіювання, характеристики}



\begin{itemize}
    \item 
\end{itemize}

\section{Аналіз успішних IT-проектів}
Існує декілька конкурентів, які реалізували механізм створення резервних копій.

Закриті та пропрієтарні проекти:
\begin{itemize}
    \item SQLShack \cite{consqlshack};
    \item IPerious -- не розробляється. 
    Копіює Oracle, SQL Server, MySQL, PostgresSQL, MariaDB. Коштує 150 евро. 
    Працює тільки на Windows \cite{coniperius};
    \item Urbackup -- створює повну копію файлової системи, а не бази данних \cite{conurbackup};
    \item AWS Backup -- підтримується AWS. Тільки EBS, RDS, DynamoDB, EFS, Storage. Тільки в AWS \cite{conawsbackup};
    \item HandyBackup -- розробляється та підтримується.
    Копіює mySQL, Postgres, maria, MSSQL, Oracle, DB2. Коштує 250 евро. Працює тільки на Windows \cite{conhandybackup}.
\end{itemize}

Відкриті проекти:
\begin{itemize}
    \item Amanda -- не підтримується. Зявляються тільки нові issues у репозиторії на платформі Github.
    Копіює тільки SQL бази данних: MySQL, PostgreSQL, MSSQL. Працює тільки на Linux \cite{conqualitynoc}.
\end{itemize}

\subsection{Аналіз відомих технічних рішень}

\subsection{Аналіз відомих програмних продуктів}

\section{Аналіз вимог до програмного забезпечення}
\subsection{Розроблення функціональних вимог}
\subsection{Розроблення нефункціональних вимог}
\subsection{Постановка комплексу завдань модулю}

\section{Висновки до розділу}


%!
% \section{Export}
% Вбудований інструмент для виконнання повної резервної копії таблиць hbase на файлову систему --- hdfs.
% Паралельне копіювання виконується завдяки праці MapReduce.

% Переваги:

% \begin{itemize}
%     \item паралельність
%     \item нативний інструмент hbase
% \end{itemize}

% Недоліки:

% \begin{itemize}
%     \item повільний, якщо мережа з низькою пропускною здатністю
%     \item потребую вільну память на hdfs для експортованих таблиць
%     \item існування проміжного пункту - hdfs
% \end{itemize}

% \section{CopyTable}
% Цей метод має такі самі переваги, як і у метода Export \cite{hbase2012backup}. CopyTable запускає також MapReduce задачу.
% Відрізняється він тим, що копіює дані з таблиць hbase не на hdfs, а одразу в іншу таблицю, 
% інший namespace, але в тому самому кластері. Одним із найголовніших недоліків даного методу є те, що CopyTable
% копіює row-by-row таблиць та виконує put для кожного з них. 
% При великих розмірах таблиці, CopyTable спричинює швидке наповнення MemStore та виклик Minor/Major Compaction.
% Такий метод не рекомендується використовувати на великих данних.

% \section{HTable API}
% Корисний тільки для Java програмістів \cite{hbase2016htable}.
% Цей метод дозволяє написати свою реалізацію бекапу.
% І одразу виникає дві проблеми:

% \begin{itemize}
%     \item програміст повинен знати як правильно бекапити
%     \item неможливість реалізувати на іншій мові програмування
% \end{itemize}

% \section{Offline backup}
% Практичний метод, який дозволяє виконати повну резервну копію інформації.
% HBase працює тільки поверх hdfs і зберігає там свої дані.
% Цей метод пропонує копіювати файлову систему hdfs і її директорію /hbase/.
% Задля забезпечення цілістних данних - hbase кластер повинен зупинити.
% У разі копіювання одного чи декількох namespaces-ів --- усі їх таблиці повинні бути вимкнуті (disable) для забезпечення persistence.
