%!TEX root = ../thesis.tex
\chapter{АНАЛІЗ ВИМОГ ДО ПРОГРАМНОГО ЗАБЕЗПЕЧЕННЯ}

\section{Загальні положення}
Кожна база данних реалізована за певною концепцією та мовою програмування.
Тому кожна з них має свій механізм створення резервних копій.

За напрямком:
\begin{itemize}
    \item реляційні;
    \item нереляційні.
\end{itemize}

За концепцією:
\begin{itemize}
    \item документний -- надає механізм для зберігання строкових файлів у базі данних, їх швидку оброботку та зручний пошук;
    \item ключ-значення (хеш-таблиці) -- як правило, використовується для кешування та service 
    discovery в інфраструктурі контейнерів, хостів;
    \item колоночний (альтернатива реляційним строковим таблицям) -- ідея заключається в тому,
    що користувачу чи аплікейшну не потрібно усі колонки, а тільки декілька з них.
    Тому у таких базах данних змінено механізм збереження записів,
    який дозволяє швидко зчитувати колонки;
    \item графовий -- абстракційна модель у вигляді графу. 
    Основна його задача - зберігати відносини між сущностями.
\end{itemize}

Існує 3 універсальних підходи до створенння бекапів:
\begin{itemize}
    \item копіювання файлової системи, на яку БД зберігає дані.
    \item 
\end{itemize}


\section{Змістовний опис і аналіз предметної області}

\section{Аналіз успішних IT-проектів}
\subsection{Аналіз відомих технічних рішень}
\subsection{Аналіз відомих програмних продуктів}

\section{Аналіз вимог до програмного забезпечення}
\subsection{Розроблення функціональних вимог}
\subsection{Розроблення нефункціональних вимог}
\subsection{Постановка комплексу завдань модулю}

\section{Висновки до розділу}


%!
% \section{Export}
% Вбудований інструмент для виконнання повної резервної копії таблиць hbase на файлову систему --- hdfs.
% Паралельне копіювання виконується завдяки праці MapReduce.

% Переваги:

% \begin{itemize}
%     \item паралельність
%     \item нативний інструмент hbase
% \end{itemize}

% Недоліки:

% \begin{itemize}
%     \item повільний, якщо мережа з низькою пропускною здатністю
%     \item потребую вільну память на hdfs для експортованих таблиць
%     \item існування проміжного пункту - hdfs
% \end{itemize}

% \section{CopyTable}
% Цей метод має такі самі переваги, як і у метода Export \cite{hbase2012backup}. CopyTable запускає також MapReduce задачу.
% Відрізняється він тим, що копіює дані з таблиць hbase не на hdfs, а одразу в іншу таблицю, 
% інший namespace, але в тому самому кластері. Одним із найголовніших недоліків даного методу є те, що CopyTable
% копіює row-by-row таблиць та виконує put для кожного з них. 
% При великих розмірах таблиці, CopyTable спричинює швидке наповнення MemStore та виклик Minor/Major Compaction.
% Такий метод не рекомендується використовувати на великих данних.

% \section{HTable API}
% Корисний тільки для Java програмістів \cite{hbase2016htable}.
% Цей метод дозволяє написати свою реалізацію бекапу.
% І одразу виникає дві проблеми:

% \begin{itemize}
%     \item програміст повинен знати як правильно бекапити
%     \item неможливість реалізувати на іншій мові програмування
% \end{itemize}

% \section{Offline backup}
% Практичний метод, який дозволяє виконати повну резервну копію інформації.
% HBase працює тільки поверх hdfs і зберігає там свої дані.
% Цей метод пропонує копіювати файлову систему hdfs і її директорію /hbase/.
% Задля забезпечення цілістних данних - hbase кластер повинен зупинити.
% У разі копіювання одного чи декількох namespaces-ів --- усі їх таблиці повинні бути вимкнуті (disable) для забезпечення persistence.
